\documentclass{article}
\usepackage[utf8]{inputenc}
\usepackage{graphicx}
\usepackage{float}
\usepackage{amsmath}
\usepackage[paper=a4paper,margin=1in]{geometry}
\usepackage[table,xcdraw]{xcolor}


\title{Image Segmentation}
\author{David Štych\\ Aleksandra Jamróz}
\date{\today{}}



\begin{document}
\maketitle

\section{Introduction}
During laboratory 3. of Audio Processing, Video Processing and Computer vision course, we faced the problem of image segmentation. Our task was to perform cells recognition on a dataset of 40 microscopic images. We received default code and then expanded it with various techniques, in order to improve its accuracy. 


\section{IoU comparision}
Indicator of our performance was IoU (intersection over union). You can see gathered results of successives techniques in the table below.  

\begin{table}[H]
\centering
\begin{tabular}{|
>{\columncolor[HTML]{FFFFFF}}c |c|c|}
\hline
\cellcolor[HTML]{C0C0C0}{\color[HTML]{000000} \textbf{Technique}} & \cellcolor[HTML]{C0C0C0}{\color[HTML]{000000}  \textbf{\begin{tabular}[c]{@{}c@{}}Performance \\ (IoU)\end{tabular}}} \\ \hline
{\color[HTML]{000000} Baseline system}                               & {\color[HTML]{000000} 0.685}                                                                                                                                                        \\ \hline
{\color[HTML]{000000} Changing gaussian filter to median}                                   & {\color[HTML]{000000} 0.694}                                                                                                                                                        \\ \hline
{\color[HTML]{000000} Changing default algorithm to watershed}                           & {\color[HTML]{000000} 0.736}                                                                                                                                                        \\ \hline
{\color[HTML]{000000} Adding dilation-erosion preprocessing}                           & {\color[HTML]{000000} 0.738}                                                                                                                                                        \\ \hline
{\color[HTML]{000000} Adding expanding}                           & {\color[HTML]{000000} 0.756}                                                                                                                                                        \\ \hline
\end{tabular}
\end{table}

In next paragraphs we will provide more details about used techniques.

\section{Preprocessing}

Default way to preprocess the images was gaussian filter. We tried exchanging gaussian to median filter and adding contrast limited adaptive histogram equalization (CLAHE).

\section{Automatic segmentation algorithm}

\section{Postprocessing}
We saw that some recognized cells are not full - that means that they contained holes. They appeared because cells had non-uniform colour. Some darker spots can be seen on their structure. We decided to try various morphological operations on generated masks. In final version we perform dilation and erosion because it resulted in best IoU. We conducted some experiments with opening and closing images, but the improvement was rather poor or there were no improvement at all.

This technique worked significantly better also after adding labels expanding to our masks. Those two techniques combined visibly improved our performance.

\section{TOADD}
- images which will take a lot of space
\end{document}





